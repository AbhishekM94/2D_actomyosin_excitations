\documentclass[12pt]{article}
%Gummi|065|=)

\newcommand\Tau{\mathrm{T}}
\title{\textbf{Active elastomer in 2D}}
\author{Abhishek M\\
		\\
		}
\date{16 April 2018}
\usepackage{amsmath}
\begin{document}

\maketitle

\section{Motivation}

The aim of this project is to extend work done on 1D active elastomer (Deb Sankar et al., 2017) to 2D and compare with experimentally observed phenomena such as pulsations and traveling waves. Using minimal phenomenological inputs, the idea is to write down force balance and conservation equations for the meshwork and force-producing molecules (myosin and actin) respectively. We will then go on to analysis of their numerical solutions and phase diagrams. 

\section{Equations}
For the case of an active elastomer, we consider the overdamped limit where the hydrodynamics of the fluid is ignored [Reference]. This gives the force balance equation,

\begin{equation}
\Gamma \dot{\textbf{u}} = \nabla \cdot (\sigma_e + \sigma_d + \sigma_a)
\end{equation} 
where $\sigma_e = \lambda(\nabla \cdot \textbf{u})I + \mu(\nabla \textbf{u} + (\nabla \textbf{u})^T) $ is the elastic stress, \textbf{u} is the displacement vector of the meshwork. The first term on the RHS indicates compression and the second term indicates pure shear. 
$\sigma_a = - \frac{\zeta_1 \rho_b }{1+\zeta_2\rho_b} \chi(\rho)\Delta\mu I$ is the form of active stress which depends on concentration of bound myosin $\rho_b$ . 
\\
\\
The dynamics of the concentration of bound myosin ($\rho_b)$ along with turnover is defined by,
 
\begin{equation} \dot{\rho_b} + \nabla \cdot(\rho_b \dot{\textbf{u}}) = D\nabla^2\rho_b + S_m \end{equation}
where $S_m = -k_{u0} e^{\alpha \epsilon}\rho_b  + k_b(1-c\epsilon) $ where $k_{u0}$ and $k_b$ are unbinding and binding rates of myosin. \emph{Currently I've not included strain dependent binding and unbinding}
\section{Weak formulation}
The above equations are solved numerically using finite element method on Fenics platform. In order to do so, the weak form (bilinear form) of these coupled equations ((1) and (2)) need to obtained. 
\\
For equation (1), this would be done as follows. The recipe is to multiply the whole equation by a test function, say $v1$, reduce it to a bilinear function, i.e, F(u,v) and integrate to obtain the solution. 
\begin{equation} 
\int_\Omega\Gamma \dot{\textbf{u}} \cdot v1 \ dx =\int_\Omega (\nabla \cdot \sigma) \cdot v1\ dx
\end{equation}
Here, $\sigma$ is the sum of all the stresses (here only $\sigma_e$ + $\sigma_a)$ and  $\Omega$ is the domain of integration. Further, RHS of (3) has to be integrated by parts as follows. 
\begin{equation}
\int_\Omega (\nabla \cdot \sigma) \cdot v1\ dx = - \int_\Omega \sigma : \nabla v1 \ dx + \int_{\delta\Omega} (\sigma \cdot n)\cdot v1 \ dx
\end{equation}
Substituting (4) in (3) and bringing all the terms to the LHS after simplification, we have 
\begin{equation}
F1 = \int_\Omega \Gamma\dot{\textbf{u}} \cdot v1 \ dx + \int_\Omega \sigma(u) : \epsilon(v1) \ - \int_{\delta\Omega} B \cdot v1 \ dx \ = 0 
\end{equation}
Since equations (1) and (2)  are coupled by $\dot{\textbf{u}}$ and the time derivatives are solved by finite difference methods, it is easier to assign an auxiliary variable \textbf{v} such that, 
$\dot{\textbf{u}} \ = \textbf{v}$ whose weak form is, 
\begin{equation} F2 = \int_\Omega\dot{\textbf{u}} \cdot v2 \ dx \ - \int_\Omega \textbf{v} \cdot v2 \ dx = 0  \end{equation} 
Further, the dynamical equation for $\rho_b$ is also written in its weak form as follows, 
\begin{equation}
F3 = \int_\Omega \dot{\rho_b} \cdot v3 \ dx \ + \int_\Omega(\nabla \cdot (\rho_b \textbf{v}))\cdot v3 \ dx - \int_\Omega -D(\nabla\rho_b \cdot\nabla v3) \ dx \ + \int_\Omega k_{u0}e^{\alpha\epsilon}\rho_b\ v3 \ dx\ - \int_\Omega k_b(1-c\epsilon)v3\ dx 
\end{equation}
 Here, v1, v2 and v3 are test functions chosen from the same function space. Now, the aim is to solve F = F1 + F2 + F3 = 0. \emph{Currently attempting to solve it with periodic boundary conditions.} 
\section{Extensions to be made}
In the above consideration,  the meshwork was isotropic (with two elastic constants). The stress tensor ($\sigma_e $) had terms indicating compression ($\epsilon_{ii}$) and pure shear ($\epsilon_{ij}$, where volume is conserved). 
\\ To increase the complexity, a simple anisotropic case can also be considered, where the stress tensor could depend on $e_1 = \frac{\epsilon_{xx} + \epsilon_{yy}}{2} , e_2 =  \frac{\epsilon_{xx} - \epsilon_{yy}}{2} \ and \ e_3 =  \frac{\epsilon_{xy} + \epsilon_{yx}}{2}  $
\section{S.I}
To validate the fenics code for equation (2),  a similar advection - diffusion equation (whose analytical solution is known) is implemented in Fenics and its error w.r.t to the analytical solution (reference) is seen.
In this reference, the authors have chosen a variable coefficient advection diffusion equation in 2D as follows, 
\\
Case (i): Without a source term
\begin{equation}
\dot{\rho_b} + \frac{\partial(u_0x\rho_b)}{\partial x} + \frac{\partial(-u_0y\rho_b)}{\partial y} = \frac{\partial}{\partial x}(D_0u_o^2x^2 \frac{\partial C}{\partial x}) + \frac{\partial}{\partial y}(D_0u_o^2y^2 \frac{\partial C}{\partial y})
\end{equation} 
This equation can be reduced to constant coefficient PDE by making the transformation, $x = exp(u_0 X) \ and \ y = exp(-u_0Y)$, whose solution is known and is, 
\begin{equation}
C(x,y,t) = \frac{1}{4\pi D_0 u_0^2t \sqrt{xyx_0 y_0}}(\frac{xy}{x_0y_0})^{(\! \frac{1}{2u_0D_0})} \ exp(\frac{-r^2 - 2(1+ D_0^2u_0^2)t^2}{4D_0t})
\end{equation}

\end{document}